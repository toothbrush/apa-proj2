\documentclass[a4paper]{article}

\usepackage{semantic}
\usepackage{dsfont}
\usepackage{array}

% things for the semantic package
\reservestyle{\command}{\textbf}
\command{let,in,:,case,of}
\mathlig{-->}{\longrightarrow}
% end semantic

\author{Jurri\"en Stutterheim\and Ruben de Gooijer \and Paul van der Walt} 
\date{\today} 
\title{Implementation notes for Type and Effect Systems}

\begin{document}

\maketitle \tableofcontents

\section{Introduction}

This project aims to implement a tool which can do a control-flow analysis on a lambda-calculus language with support for
data structures. The focus is on lists and pairs, for now.
%todo: more. 

\section{Design}

Here we will detail the design of the language, for example which features it
supports. In Tables \ref{tab:elems} and \ref{tab:grammar} one can see the
grammar of the language. It is based on Haskell (in fact we use the Haskell
source extensions parser), but of course only a subset is supported. 

We include support for pairs and lists, as well as a case statement in which one
can pattern-match on these data structures. There is not yet support for general
data structures.

\subsection{A simple functional language}

%todo: copied from slides, may need adjustment.

The basic elements of the language can be found in Table \ref{tab:elems}. The
language includes natural number constants, variables, and operators. Program
points are introduced whenever a lambda-expression is encountered.
\begin{table}
    \centering
    \begin{tabular}{rcll}
        $n$ &           $\in$ & \textbf{Num = $\mathds{N}$}& numerals \\
        $f,x$ &         $\in$ & \textbf{Var}               & variables \\
        $\oplus$ &      $\in$ & \textbf{Op}                & binary operators \\
        $\pi$ &         $\in$ & \textbf{Pnt}               & program points  \\
        $t$ &           $\in$ & \textbf{Tm}                & terms \\
    \end{tabular}
    \caption{Basic elements}
    \label{tab:elems}
\end{table}

In Table \ref{tab:grammar} one can see what constructs are valid in our subset
of Haskell, since only a very limited subset is supported.

\begin{table}
    \centering
    \begin{tabular}{lcl}
        \hline
        $t$     & ::= & $n\: |\: \textbf{ false }\: |\: \textbf{ true }\: |\: x\: |\: \lambda_\pi x.t_1\: |\: \lambda_\pi x . t_1$ \\
                & $|$ & $t_1 t_2 \:|\: \textbf{ if } t_1 \textbf{ then } t_2 \textbf{ else } t_3 \:|\:  \textbf{ let } x = t_1 \textbf{ in } t_2$\\
                & $|$ & $\mu f.\textbf{ let } x = t_1 \textbf{ in } t_2$\\
                & $|$ & $t_1 \oplus t_2 \:|\: [\,]_\pi \:|\:t_1\<:>_\pi t_2 \: | \: (t_1,t_2)_\pi $ \\ 
                & $|$ & $ \textbf{case } e_0 \textbf{ of } alts $ \\
        $alts$  & ::= & $alt_1; \ldots ; alt_n$ \\
        $alt$   & ::= & $patt\rightarrow t$ \\
        $patt$  & ::= & $x \:|\: (patt, patt)$ \\
                & $|$ & $(x: patt)  \:|\: [\,]$\\
                & $|$ & $\_$\\

        \hline
    \end{tabular}
    \caption{Grammar. $t$ is a term.}
    \label{tab:grammar}
\end{table}

\subsection{Type System}

Here we will detail the type system which has been implemented in this project.
It consists of 

\begin{table}
    \centering
    \begin{tabular}{rcll}
        $\varphi$ &               $\in$ & \textbf{Ann}                   & annotations \\ 
        $\widehat{\tau}$&         $\in$ & $\widehat{\textbf{Ty}      } $ & annotated types \\
        $\widehat{\sigma} $&      $\in$ & $\widehat{\textbf{TyScheme}} $ & annotated type schemes\\
        $\widehat{\Gamma}$&       $\in$ & $\widehat{\textbf{TyEnv}   } $ & annotated type environments  \\
    \end{tabular}
    \caption{Typing elements, annotated types}
    \label{tab:typingelems}
\end{table}
\begin{table}
    \centering
    \begin{tabular}{lcl}
        $ \varphi$         & ::= & $ \emptyset \:|\: \{\pi\} \:|\: \varphi_1 \cup \varphi_2 $ \\
        $\widehat{\tau}$   & ::= & $\alpha \:|\: Nat \: | \: Bool \: | \: \widehat{\tau}_1 \stackrel{\varphi}{\rightarrow} \widehat{\tau}_2 \:|\: \widehat{\tau}_1 \times^\varphi \widehat{\tau}_2 \:|\: \widehat{\tau}~\texttt{list}^\varphi   $ \\ 
        $\widehat{\sigma}$ & ::= & $\widehat{\tau} \:|\: \forall \alpha. \widehat{\sigma}_1 $ \\ 
        $\widehat{\Gamma}$ & ::= & $[\,] \:|\: \widehat{\Gamma}_1[x \mapsto \widehat{\sigma}] $ \\
    \end{tabular}
    \caption{Typing}
    \label{tab:typing}
\end{table}

\subsection{Natural semantics}

In this section, Table \ref{tab:natural-semantics}, we define how terms are
evaluated. This is also known as big-step semantics. Notice the addition of
\textbf{case}-statements for data structure use. See Table 5.4 in \cite{nnh}
for reference. Our listing isn't complete, only the additions with respect to the 
aforementioned table are shown. 

\begin{table}
    \centering
    \begin{tabular}{ll}
        \hline
        $ [$\emph{ns-con}$] $& $ |- c --> c$ \\ ~&~\\
        $ [$\emph{ns-fn}$]  $& $ |- \lambda_\pi x.t_1 --> \lambda_\pi x.t_1$ \\ ~&~\\
        $ [$\emph{ns-fn$_{rec}$}$]  $& $ |- \mu f.\lambda_\pi x.t_1 --> \lambda_\pi x.(t_1[f \mapsto \lambda_\pi x.t_1 ])$ \\ ~&~\\
        $ [$\emph{ns-case$_{pair}$}$] $& \inference{|- e_0 --> (x_1,x_2) & |- x_1 --> v_1 & |- x_2 --> v_2}
{|- (\<case>\: e_0\: \<of>\: (x_1,x_2) -> e_1) --> e_1[x_1\mapsto v_1, x_2 \mapsto v_2]} \\ ~&~\\
        $ [$\emph{ns-case$_{list1}$}$] $& \inference{|- e_0 --> [] }
{|- (\<case>\: e_0\: \<of>\: [] -> e_1; (x\<:>xs) -> e_2) --> e_1} \\ ~&~\\
        $ [$\emph{ns-case$_{list2}$}$] $& \inference{|- e_0 --> (x\<:>xs) & x --> v & xs --> vs }
{|- (\<case>\: e_0\: \<of>\: [] -> e_1; (x\<:>xs) -> e_2) --> e_2[x\mapsto v, xs \mapsto vs]} \\ ~&~\\


        \hline
    \end{tabular}
    \caption{Natural semantics for the language}
    \label{tab:natural-semantics}
\end{table}

\subsection{Control Flow Analysis}

We have the usual set of rules including, for example,  \emph{cfa-var} or
\emph{cfa-let}, which has been extended to support lists, see \emph{cfa-nil}
and \emph{cfa-cons}, the rules specifying list-typing, Table
\ref{tab:cfa-rules}. CFA judgements look like {$\widehat{\Gamma}\:\vdash_{CFA}
t : \widehat{\sigma}$}, meaning that some term $t$ has inferred type
$\widehat{\sigma}$ in context $\widehat{\Gamma}$. See Table 5.2 in \cite{nnh}
for reference. 


\begin{table}
    \centering
    \begin{tabular}{ll}
        \hline
        $ [$\emph{cfa-var}$] $& \inference{\widehat{\Gamma}(x) = \widehat{\sigma}}
{\widehat{\Gamma} |-_{CFA} x  : \widehat{\sigma}} \\
~&~\\
$[$\emph{cfa-let}$] $& \inference{\widehat{\Gamma} |-_{CFA} t_1:\widehat{\sigma}_1 
& \widehat{\Gamma}[x \mapsto \widehat{\sigma}_1] |-_{CFA} t_2 : \widehat{\tau}}
{\widehat{\Gamma} |-_{CFA}\: \<let>\: x = t_1\: \<in>\: t_2 : \widehat{\tau}} \\
~&~\\
$[$\emph{cfa-nil}$] $& 
\inference{}
{\widehat{\Gamma} |-_{CFA} [\,] : [\widehat{\sigma}]} \\
~&~\\
$[$\emph{cfa-cons}$] $& \inference{\widehat{\Gamma} |-_{CFA} t_1:\widehat{\sigma} 
 \widehat{\Gamma}|-_{CFA} t_2 : [\widehat{\sigma}]}
{\widehat{\Gamma} |-_{CFA} t_1 \<:> t_2 :  [\widehat{\sigma}]} \\
~&~\\
$[$\emph{cfa-pair}$] $& \inference{\widehat{\Gamma} |-_{CFA} t_1 : \widehat{\tau}_1 &  \widehat{\Gamma} |-_{CFA} t_2 : \widehat{\tau}_2   } 
{\widehat{\Gamma} |-_{CFA} (t_1,t_2) : \widehat{\tau}_1 \times^\varphi \widehat{\tau}_2 } \\
~&~\\
$[$\emph{cfa-case-pair}$] $& \inference{\widehat{\Gamma} |-_{CFA} e_0 : \widehat{\tau}_1 \times^\varphi \widehat{\tau}_2 
&  \widehat{\Gamma} |-_{CFA} e_1 : \widehat{\tau}   } 
{\widehat{\Gamma} |-_{CFA} \:\<case>\: e_0\: \<of>\: (x_1,x_2) \Rightarrow e_1 : \widehat{\tau} } \\
~&~\\
$[$\emph{cfa-case-nil}$] $& \inference{\widehat{\Gamma} |-_{CFA} e_0 : [\widehat{\tau}_1] &  \widehat{\Gamma} |-_{CFA} e_1 : \widehat{\tau}   } 
{\widehat{\Gamma} |-_{CFA} \:\<case>\: e_0\: \<of>\: [] \Rightarrow e_1 : \widehat{\tau} }, if $e_0$ empty\\
~&~\\
$[$\emph{cfa-case-cons}$] $& \inference{\widehat{\Gamma} |-_{CFA} e_0 : [\widehat{\tau}_1] &  \widehat{\Gamma} |-_{CFA} e_1 : \widehat{\tau}   } 
{\widehat{\Gamma} |-_{CFA} \:\<case>\: e_0\: \<of>\: (x\<:>xs) \Rightarrow e_1 : \widehat{\tau} }, if $e_0$ nonempty\\

        \hline
    \end{tabular}
    \caption{Control Flow Analysis rules. }
    \label{tab:cfa-rules}
\end{table}

In Table \ref{tab:algoU} we see the unification algorithm used for CFA. If none
of the cases match, the algorithm fails, which is correct behaviour. The last
rule, for lists, was added to the default implementation.

\begin{table}
    \centering
    \begin{tabular}{rcb{7cm}}
        \hline
        $\mathcal{W}_{CFA}(\widehat{\Gamma},[])$ & = & $(\tau\: \texttt{list}^{\alpha}, id, \emptyset)$ with $\tau,\alpha$ fresh \\
        $\mathcal{W}_{CFA}(\widehat{\Gamma},(x\<:>xs))$ & = &  $ 
\begin{array}{lll}
    let & \beta & \textnormal{fresh annotation variable}\\
        & (listTy,\theta_1,c_1) & = \mathcal{W}_{CFA}(\widehat{\Gamma}, x)\\
    & (\sigma, \theta_2, c_2) & = \mathcal{W}_{CFA}(\theta_1 \widehat{\Gamma}, xs)\\
    & \theta_3 & = \mathcal{U}_{CFA} (\sigma, \theta_2\: listTy) \\ 
        & \theta&  =  \theta_3 \circ \theta_2 \circ \theta_1\\ 
        & c_3 & = \beta \leq listTy \cup \beta \leq \sigma \\ 
        & cs & = c_1 \cup c_2 \cup c_3 \\
         in & ~  & (\theta\: (listTy\: \texttt{list}^\beta), \theta, \theta\: cs)
\end{array}
 $  
\\




        \hline
    \end{tabular}
    \caption{Algorithm~W for lambda calculus including data structures. }
    \label{tab:algoW}
\end{table}
\begin{table}
    \centering
    \begin{tabular}{rcb{7cm}}
        \hline
        $\mathcal{U}_{CFA}(Nat,Nat)$ & = & $id$ \\
        $\mathcal{U}_{CFA}(Bool,Bool)$ & = & $id$ \\
        $\mathcal{U}_{CFA}(\tau_1,\tau_2)$ & = & \parbox{0.3\textwidth}{$ \left\{
\begin{array}{ll}
    id                      & if \tau_1 = \tau_2         \\
    \left[ \tau_1 \mapsto \tau_2\right] & if \tau_1 \notin fv(\tau_2)\\
    \left[ \tau_2 \mapsto \tau_1\right] & if \tau_2 \notin fv(\tau_1)\\
    fail& otherwise
\end{array}
\right. $ } \\
        $\mathcal{U}_{CFA}(\beta,\tau)$ & = & \parbox{\textwidth}{$ \left\{
\begin{array}{ll}
    \left[ \tau \mapsto \beta\right] & if \tau \notin fv(\beta)\\
    occurs check & otherwise
\end{array}
\right. $ } \\
        $\mathcal{U}_{CFA}(\tau,\beta)$ & = & $\mathcal{U}_{CFA}(\beta,\tau)$ \\
        $\mathcal{U}_{CFA}(\tau_1 \stackrel{\varphi}{->} \tau_2,
                           \tau_1' \stackrel{\varphi'}{->} \tau_2')$ & = & 
                           \parbox{\textwidth}{$ 
\begin{array}{ll}
    let & \theta_1 = \mathcal{U}'_{CFA}(\varphi,\varphi')  \\
        & \theta_2 = \mathcal{U}_{CFA}(\theta_1 \tau_1, \theta_1 \tau_1') \\
        & \theta_3 = \mathcal{U}_{CFA}(\theta_2 (\theta_1 \tau_2), \theta_2(\theta_1 \tau_2')) \\
    in  & \theta_3 \circ \theta_2 \circ \theta_1
\end{array}
 $ } 
\\
        $\mathcal{U}_{CFA}([\tau]^\varphi,
        [\tau']^{\varphi'})$ & = & 
                           \parbox{0.3\textwidth}{$ 
\begin{array}{ll}
    let & \theta_1 = \mathcal{U}_{CFA}(\tau,\tau')  \\
        & \theta_2 = \mathcal{U}'_{CFA}(\theta_1 \varphi, \theta_1 \varphi') \\
    in  & \theta_2 \circ \theta_1
\end{array}
 $ } 
\\

        $\mathcal{U}'_{CFA}(\varphi,\varphi')$ & = & \parbox{\textwidth}{$ \left\{
\begin{array}{ll}
    id & if \varphi = \varphi' \\
    \left[ \varphi' \mapsto \varphi\right] & otherwise 
\end{array}
\right. $ } \\
        $\mathcal{U}'_{CFA}(\_,\_)$ & = & unification error \\
        \hline
    \end{tabular}
    \caption{Unification algorithm for lambda calculus including data structures. $\mathcal{U}'$ is 
    annotation unification.}
    \label{tab:algoU}
\end{table}
\section{Approach}


\section{Implemented}

\section{Example programs}

%todo: this probably needs fixing. 
\begin{itemize}
    \item \texttt{app.hm} An illustration of the function application in action. 
    \item \texttt{assignment.hm} The example program from the assignment, with
        the correct output. 
    \item \texttt{identity.hm} The simplest possible program, the identity
        function. 
    \item \texttt{inf.hm} Gives a type error, correctly (the program, $\lambda
        x.
        \lambda y . (y x) (x y)$ isn't typable and leads to a so-called
        \emph{occurs check\footnote{Meaning that the unification algorithm would
        have to introduce an infinite type to succeed.}}).
    \item \textit{singleton.hm} Simply illustrates a variable which is out of
        scope. 
\end{itemize}



\begin{thebibliography}{9}

\bibitem{nnh}
  {Flemming Nielson, Hanne Riis Nielson, Chris Hankin},
  \emph{Principles of Program Analysis},
  Springer, Berlin,
  2nd Edition,
  2005.

\end{thebibliography}

\end{document}
