
%outline:

% an explanation of both the type system, a run through of one or two examples 
% an architectural description of the implementation.
% Features and limitations should be prominently displayed in your documentation. 

\documentclass[a4paper]{article}

\usepackage{semantic}
\usepackage{dsfont}
\usepackage{a4wide}
\usepackage{array}
\usepackage{bussproofs}
\usepackage{latexsym}

% things for the semantic package
\reservestyle{\command}{\textbf}
\command{let,in,:,case,of,if,then,else}
\mathlig{-->}{\longrightarrow}
\newcommand{\tyrel}{\sqsubseteq}
% end semantic

\author{Jurri\"en Stutterheim\footnote{3555003}\and Ruben de Gooijer\footnote{3508617}\and Paul van der Walt\footnote{3120805}}
\date{\today} 
\title{Implementation notes for Type and Effect Systems}

\begin{document}

\maketitle \tableofcontents

\section{Introduction}

This project aims to implement a tool which can do a control-flow analysis on a
lambda-calculus language with support for data structures. The focus is on
lists%and pairs
, for now.

Control-flow analysis aims to determine for each function application, which
functions may be applied. We also track the creation of data structures and that
of lambda functions. The implementation of our analysis is built on the
algorithm W type inferencing algorithm, using an annotated type system.

Our implementation currently offers support for monovariance and subeffecting.
While support for polyvariance has been attempted, we failed to deliver it fully
operational on time.

This document is organised as follows. First we introduce the design of the
analysis by showing our target language, its type system, its semantics and
our analysis rules. We then proceed to the implementation details of the
analysis.

\section{Design}

Here we will detail the design of the language, for example which features it
supports. In Tables \ref{tab:elems} and \ref{tab:grammar} one can see the
grammar of the language. It is based on Haskell (in fact we use the Haskell
source extensions parser), but of course only a subset is supported. 

We include support for% pairs and
 lists, as well as a case statement in which one
can pattern-match on these data structures. There is not yet support for general
data structures.

\subsection{A simple functional language}

%todo: copied from slides, may need adjustment.

The basic elements of the language can be found in Table \ref{tab:elems}. The
language includes natural number constants, variables, and operators. Program
points are introduced whenever a lambda-expression is encountered.
\begin{table}
    \begin{centering}
    \begin{tabular}{rcll}
        $n$ &           $\in$ & \textbf{Num = $\mathds{N}$}& numerals \\
        $f,x$ &         $\in$ & \textbf{Var}               & variables \\
        $\oplus$ &      $\in$ & \textbf{Op}                & binary operators \\
        $\pi$ &         $\in$ & \textbf{Pnt}               & program points  \\
        $t$ &           $\in$ & \textbf{Tm}                & terms \\
    \end{tabular}
    \caption{\label{tab:elems}Basic elements}
    \end{centering}
\end{table}

In Table \ref{tab:grammar} one can see what constructs are valid in our subset
of Haskell, since only a very limited subset is supported. Note that recursion
is only supported inside \textbf{let}s. 

\begin{table}
    \begin{centering}
    \begin{tabular}{lcl}
        \hline
        $t$    & ::= & $n\: |\: \textbf{ false }\: |\: \textbf{ true }\: |\: x\: |\: \lambda_\pi x.t_1 \:$ \\
               & $|$ & $t_1 t_2 \:|\: \textbf{ if } t_1 \textbf{ then } t_2 \textbf{ else } t_3 \:|\:  \textbf{ let } x = t_1 \textbf{ in } t_2$\\
               & $|$ & $\mu f.\textbf{ let } x = t_1 \textbf{ in } t_2$\\
               & $|$ & $t_1 \oplus t_2 \:|\: [\,]_\pi \:|\:t_1\<:>_\pi t_2   $\\%     \: | \: (t_1,t_2)_\pi $ \\ 
               & $|$ & $ \textbf{case } e_0 \textbf{ of } alts $ \\
        $alts$ & ::= & $alt_0; alt_1$ \\
        $alt$  & ::= & $patt \rightarrow t$ \\
        $patt$ & ::= & $x \:|\: \textbf{True} \:|\: \textbf{False}            $\\%                \:|\: (patt, patt)$ \\
               & $|$ & $(x: patt)  \:|\: [\,]$\\
        \hline
    \end{tabular}
    \caption{Abstract syntax. $t$ is a term.}
    \label{tab:grammar}
    \end{centering}
\end{table}

\subsection{Type System}

Here we will detail the type system which has been implemented in this project.
It consists of a polymorphic type system, with the important addition of
annotations. We use annotations and constraints to be able to do a control-flow
analysis, eventually telling us which functions one may expect to be used where. 

Our type system has variables, arrows, and type environments, as well as
annotations, which is what makes it different from a normal polymorphic type
system. Table \ref{tab:typingelems} and \ref{tab:typing} show the elements of
the type system. 

\begin{table}
    \begin{centering}
    \begin{tabular}{rcll}
        $\varphi$ &               $\in$ & \textbf{Ann}                   & annotations \\ 
        $\widehat{\tau}$&         $\in$ & $\widehat{\textbf{Ty}      } $ & annotated types \\
        $\widehat{\sigma} $&      $\in$ & $\widehat{\textbf{TyScheme}} $ & annotated type schemes\\
        $\widehat{\Gamma}$&       $\in$ & $\widehat{\textbf{TyEnv}   } $ & annotated type environments  \\
    \end{tabular}
    \caption{Typing elements, annotated types}
    \label{tab:typingelems}
    \end{centering}
\end{table}
\begin{table}
    \begin{centering}
    \begin{tabular}{lcl}
        $ \varphi$         & ::= & $ \emptyset \:|\: \{\pi\} \:|\: \varphi_1 \cup \varphi_2 $ \\
        $\widehat{\tau}$   & ::= & $\alpha \:|\: Nat \: | \: Bool \: | \: $ \\
        & $|$ & $\widehat{\tau}_1^{\varphi_1}
                           \stackrel{\varphi}{\rightarrow} 
                           \widehat{\tau}_2^{\varphi_2} $ \\
                           & $|$ & $ % \widehat{\tau}_1 \times^\varphi \widehat{\tau}_2 \:|\: 
                           [\widehat{\tau}]^\varphi   $ \\ 
        $\widehat{\sigma}$ & ::= & $\widehat{\tau} \:|\: \forall \alpha. \widehat{\sigma}_1 $ \\ 
        $\widehat{\Gamma}$ & ::= & $[\,] \:|\: \widehat{\Gamma}_1[x \mapsto \widehat{\sigma}] $ \\
    \end{tabular}
    \caption{Typing}
    \label{tab:typing}
    \end{centering}
\end{table}

Next a list of typing judgements will be given. This constitutes the final type
system. %See weijers fig 5.3 for a reference. 

These rules are formulated in a syntax-directed fashion. 

\begin{table}
    \begin{centering}
    \begin{tabular}{ll}
        \hline \\
        $ [$\emph{t-nat}$] $& \inference{}
        {
        {\Gamma},C |- \mathds{N} : Nat^{\varphi}
        } \\
~&~\\
        $ [$\emph{t-true}$] $& \inference{}
        {
        {\Gamma},C |- True : Bool^{\varphi}
        } \\
~&~\\
        $ [$\emph{t-false}$] $& \inference{}
        {
        {\Gamma},C |- False : Bool^{\varphi}
        } \\
~&~\\
        $ [$\emph{t-nil}$] $& \inference{
        {\Gamma},C |- Nil : \tau^{\varphi_1}
        & C |- \varphi \sqsupseteq \pi
        }
        {
        {\Gamma},C |- Nil_\pi : [\tau^{\varphi_1}]^{\varphi}
        } \\
~&~\\
        $ [$\emph{t-cons}$] $& \inference{
        {\Gamma},C |- e_1 : \tau^{\varphi_1}
        & \Gamma,C |- e_2 : [\tau^{\varphi_2}]^{\varphi_3}
        & C |- \varphi_1 \sqsupseteq \varphi_2
        & C |- \varphi \sqsupseteq \pi
        }
        {
        {\Gamma},C |- Cons_\pi\ e_1\ e_2  : [\tau^{\varphi_1}]^{\varphi}
        } \\
~&~\\
        $ [$\emph{t-var}$] $& \inference{
        {\Gamma}(x) = ({\sigma},\varphi)
        & C |- \varphi_1 \sqsupseteq \varphi
        & inst(\sigma) = \tau
        }
        {
        {\Gamma},C |- x  : {\tau^{\varphi_1}}
        } \\
~&~\\
        $ [$\emph{t-app}$] $& \inference{
        \Gamma,C |- e_1 : \tau_2^{\varphi_2} ->^{\varphi} \tau_0^{\varphi_0}
        & \Gamma,C|- e_2 : \tau_2^{\varphi_3}
        & C|- \varphi_2 \sqsupseteq \varphi_3
        & C|- \varphi_4 \sqsupseteq \varphi_0
        }
        {
        \Gamma, C |- e_1 e_2 : \tau_0^{\varphi_4}
        } \\
~&~\\
        $ [$\emph{t-if}$] $& \inference{
        \Gamma, C |- e_0 : Bool^{\varphi_0}
        & \Gamma, C |- e_1 : \tau^{\varphi_1}
        & \Gamma, C |- e_2 : \tau^{\varphi_2}
        & C |- \varphi_0 \tyrel \varphi
        & C |- \varphi_1 \tyrel \varphi
        & C |- \varphi_2 \tyrel \varphi
        }
        {
        \Gamma,C|- \<if>\: e_0\: \<then>\: e_1\: \<else>\: e_2 : \tau^{\varphi}
        } \\
~&~\\
        $ [$\emph{t-bin-op}$] $& \inference{ % what about all the annotated simple vars?
        \Gamma, C |- e_1 : \tau_{1\oplus}^{\varphi_1}
        & \Gamma, C |- e_2 : \tau_{2\oplus}^{\varphi_2}
        & C |- \varphi_1 \tyrel \varphi
        & C |- \varphi_2 \tyrel \varphi
        }
        {
        \Gamma, C|- e_1 \oplus e_2 : \tau_{\oplus}^\varphi
        } \\
~&~\\
% todo do we need rules for case statements involving head and tail of lists??
        $ [$\emph{t-let}$] $& \inference{
        \Gamma, C' |- e_1 : \tau_1^{\varphi_1}
        & (C'',\sigma) = gen(C', \tau_1)
        & \Gamma[x \mapsto (\sigma, \varphi_1)],C\cup C'' |- e_2 : \tau_2^{\varphi_2}
        & C\cup C'' |- \varphi_2 \tyrel \varphi_3
        }
        {
        \Gamma,C\cup C'' |- \<let>\: x = e_1\: \<in>\: e_2 : \tau_2^{\varphi_3}
        } \\
~&~\\
%         $ [$\emph{t-app}$] $& \inference{
% ~ 
%         }
%         {
% ~ 
%         } \\
% ~&~\\
%TODO FIXME 
        \hline
    \end{tabular}
    \caption{Typing judgements}
    \label{tab:typing-rules}
    \end{centering}
\end{table}

\subsection{Example Derivation}

\begin{prooftree}
\small
\AxiomC{$asdf$}
\RightLabel{\scriptsize{id}}
\UnaryInfC{$A a \vdash A a$}
  \AxiomC{$instance A a \Rightarrow A (Maybe a)$}
  \RightLabel{\scriptsize{inst}}
\BinaryInfC{$A a oplus A Maybe( a)$}
\RightLabel{\scriptsize{closure}}
\UnaryInfC{$A (Maybe a) oplus A (Maybe (Maybe a))$}
  \AxiomC{$$}
  \RightLabel{\scriptsize{id}} 
  \UnaryInfC{$A a oplus A a$}
    \AxiomC{$instance A a \Rightarrow A (Maybe a)$} 
  \RightLabel{\scriptsize{inst}}
  \BinaryInfC{$A a oplus A (Maybe a)$}
\RightLabel{\scriptsize{trans}}
\BinaryInfC{$A  a oplus A (Maybe (Maybe a))$}
\RightLabel{\scriptsize{closure}}
\UnaryInfC{$A Int oplus A (Maybe (Maybe Int))$}
  \AxiomC{$asdf$}
  \RightLabel{\scriptsize{id}}
  \UnaryInfC{$B a oplus B a$}
    \AxiomC{$class (A a) \Rightarrow B a$}
  \BinaryInfC{$B a oplus A a$}
  \RightLabel{\scriptsize{closure}}
  \UnaryInfC{$B Int oplus A Int$}
\RightLabel{\scriptsize{trans}}
\BinaryInfC{$B Int oplus A (Maybe (Maybe Int))$}
\end{prooftree}


\subsection{Natural semantics}

In this section, Table \ref{tab:natural-semantics}, we define how terms are
evaluated. This is also known as big-step semantics. Notice the addition of
\textbf{case}-statements for data structure use. See Table 5.4 in \cite{nnh}
for reference. Our listing isn't complete, only the additions with respect to the 
aforementioned table are shown. 

\begin{table}
    \begin{centering}
    \begin{tabular}{ll}
        \hline
        $ [$\emph{ns-con}$] $& $ |- c --> c$ \\ ~&~\\
        $ [$\emph{ns-fn}$]  $& $ |- \lambda_\pi x.t_1 --> \lambda_\pi x.t_1$ \\ ~&~\\
%        $ [$\emph{ns-fn$_{rec}$}$]  $& $ |- \mu f.\lambda_\pi x.t_1 --> \lambda_\pi x.(t_1[f \mapsto \lambda_\pi x.t_1 ])$ \\ ~&~\\
%         $ [$\emph{ns-case$_{pair}$}$] $& \inference{|- e_0 --> (x_1,x_2) & |- x_1 --> v_1 & |- x_2 --> v_2}
% {|- (\<case>\: e_0\: \<of>\: (x_1,x_2) -> e_1) --> e_1[x_1\mapsto v_1, x_2 \mapsto v_2]} \\ ~&~\\
        $ [$\emph{ns-case$_{list1}$}$] $& \inference{|- e_0 --> [] }
{|- (\<case>\: e_0\: \<of>\: [] -> e_1; (x\<:>xs) -> e_2) --> e_1} \\ ~&~\\
        $ [$\emph{ns-case$_{list2}$}$] $& \inference{|- e_0 --> (x\<:>xs) & x --> v & xs --> vs }
{|- (\<case>\: e_0\: \<of>\: [] -> e_1; (x\<:>xs) -> e_2) --> e_2[x\mapsto v, xs \mapsto vs]} \\ ~&~\\


        \hline
    \end{tabular}
    \caption{Natural semantics for the language}
    \label{tab:natural-semantics}
    \end{centering}
\end{table}

\subsection{Control Flow Analysis}

We have the usual set of rules including, for example,  \emph{cfa-var} or
\emph{cfa-let}, which has been extended to support lists, see \emph{cfa-nil}
and \emph{cfa-cons}, the rules specifying list-typing, Table
\ref{tab:cfa-rules}. CFA judgements look like {$\widehat{\Gamma}\:\vdash_{CFA}
t : \widehat{\sigma}$}, meaning that some term $t$ has inferred type
$\widehat{\sigma}$ in context $\widehat{\Gamma}$. See Table 5.2 in \cite{nnh}
for reference. 


\begin{table}
    \begin{centering}
    \begin{tabular}{ll}
        \hline
        $ [$\emph{cfa-var}$] $& \inference{\widehat{\Gamma}(x) = \widehat{\sigma}}
{\widehat{\Gamma} |-_{CFA} x  : \widehat{\sigma}} \\
~&~\\
$[$\emph{cfa-let}$] $& \inference{\widehat{\Gamma} |-_{CFA} t_1:\widehat{\sigma}_1 
& \widehat{\Gamma}[x \mapsto \widehat{\sigma}_1] |-_{CFA} t_2 : \widehat{\tau}}
{\widehat{\Gamma} |-_{CFA}\: \<let>\: x = t_1\: \<in>\: t_2 : \widehat{\tau}} \\
~&~\\
$[$\emph{cfa-nil}$] $& 
\inference{}
{\widehat{\Gamma} |-_{CFA} [\,] : [\widehat{\sigma}]} \\
~&~\\
$[$\emph{cfa-cons}$] $& \inference{\widehat{\Gamma} |-_{CFA} t_1:\widehat{\sigma} 
 \widehat{\Gamma}|-_{CFA} t_2 : [\widehat{\sigma}]}
{\widehat{\Gamma} |-_{CFA} t_1 \<:> t_2 :  [\widehat{\sigma}]} \\
~&~\\
% $[$\emph{cfa-pair}$] $& \inference{\widehat{\Gamma} |-_{CFA} t_1 : \widehat{\tau}_1 &  \widehat{\Gamma} |-_{CFA} t_2 : \widehat{\tau}_2   } 
% {\widehat{\Gamma} |-_{CFA} (t_1,t_2) : \widehat{\tau}_1 \times^\varphi \widehat{\tau}_2 } \\
% ~&~\\
% $[$\emph{cfa-case-pair}$] $& \inference{\widehat{\Gamma} |-_{CFA} e_0 : \widehat{\tau}_1 \times^\varphi \widehat{\tau}_2 
% &  \widehat{\Gamma} |-_{CFA} e_1 : \widehat{\tau}   } 
% {\widehat{\Gamma} |-_{CFA} \:\<case>\: e_0\: \<of>\: (x_1,x_2) \Rightarrow e_1 : \widehat{\tau} } \\
% ~&~\\
$[$\emph{cfa-case-nil}$] $& \inference{\widehat{\Gamma} |-_{CFA} e_0 : [\widehat{\tau}_1] &  \widehat{\Gamma} |-_{CFA} e_1 : \widehat{\tau}   } 
{\widehat{\Gamma} |-_{CFA} \:\<case>\: e_0\: \<of>\: [] \Rightarrow e_1 : \widehat{\tau} }, if $e_0$ empty\\
~&~\\
$[$\emph{cfa-case-cons}$] $& \inference{\widehat{\Gamma} |-_{CFA} e_0 : [\widehat{\tau}_1] &  \widehat{\Gamma} |-_{CFA} e_1 : \widehat{\tau}   } 
{\widehat{\Gamma} |-_{CFA} \:\<case>\: e_0\: \<of>\: (x\<:>xs) \Rightarrow e_1 : \widehat{\tau} }, if $e_0$ nonempty\\

        \hline
    \end{tabular}
    \caption{Control Flow Analysis rules. }
    \label{tab:cfa-rules}
    \end{centering}
\end{table}


\section{Program design}
The program compiles to a single executable, named \emph{cfa}. It relies heavily
on the UU Attribute Grammar system. Below is a list with filenames and a short
description what every one of these files does.

\begin{description}
\item [Components.hs] Uses haskell-src-exts to parse Haskell code and convert
it into a simplified AST. It also provides functions to start the analysis.
\item [Main.hs] Main file containing the program entry point.
\item [AG.ag] Main AG file which includes all other AG files. Contains the worklist,
unification and substitution algorithms.
\item [CollectBinders.ag] Disects a quantified type into separate parts.
\item [DataTypes.ag] Contains all of our data types.
\item [FreeAnnVars.ag] Collects all free annotation variables from the AST.
\item [FreeTyVars.ag] Collects all free type variables from the AST.
\item [FreeVars.ag] Collects all free variables from the AST.
\item [Infer.ag] Contains algorithm W and the generalise and instantiate functions.
\end{description}

\section{Approach}


\section{Features and Limitations}

The assignment suggested implementing CFA with support for tracking data structure 
creation, with lists, pairs and a case-statement. Our implementation supports lists and 
a case statement, with pattern matching and a fall-through case. 

% ???
The implementation so far only has support for 

\section{Example programs}

Example programs can be found in the \texttt{examples/} directory of the distribution, 
and have an extension \texttt{.hm}. Feed the program the files via \emph{stdin}. Each file
includes comments which describe its expected analysis result, and possible peculiarities
it may have. Note the directory \texttt{ill-typed/}, in which the examples fail by design.
They include examples of infinite types, etc. 


\begin{thebibliography}{9}

\bibitem{nnh}
  {Flemming Nielson, Hanne Riis Nielson, Chris Hankin},
  \emph{Principles of Program Analysis},
  Springer, Berlin,
  2nd Edition,
  2005.

\end{thebibliography}

\end{document}
